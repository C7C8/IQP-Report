\documentclass[12pt]{article}                         % Make it an article
\usepackage{indentfirst}                       % All paragraphs indented
\usepackage{setspace}                           % For 1.5 spacing
\usepackage[hidelinks]{hyperref}      % Make table of contents clickable
\usepackage[margin=1in]{geometry} % 1 inch margin
\usepackage[english]{babel}                     % For "fancy" quotes
\usepackage[autostyle, english=american]{csquotes} % Same
\usepackage{apacite}                            % Use APA citations
\usepackage{etoolbox}                           % Misc
\usepackage{titlesec}
\usepackage{listofauthorships}
\MakeOuterQuote{"}                              % For fancy quotes
\usepackage{times}                              %For Tilde
\usepackage{textcomp}                           %For Tilde
\titlespacing*{\section}{0pt}{0pt}{0pt}
\titlespacing*{\subsection}{0pt}{0pt}{0pt}
\titlespacing*{\subsubsection}{0pt}{5pt}{-3pt}

\title{Eilat Transportation}
\author{Evan Goldstein, Valeria Kopper, Christopher Myers, Zachary Zlotnick}
\date{January 2018}

\begin{document}
\pagenumbering{gobble}
\maketitle
\newpage

\renewcommand\abstractname{Summary} % "Abstract" -> "Summary", the header lies

\pagenumbering{roman}
\tableofcontents
\newpage
\listofauthorships
\newpage
\pagenumbering{arabic}
\doublespacing

% Actual content begins here!
\section{Introduction}

Goal: The goal of this project is to aid the city of Eilat in investigating possible transportation services that will minimize environmental impact, congestion, and travel times for residents, tourists, and businesses while preserving travel friendliness and experience.

Objectives:
\begin{itemize}
    \item Develop an understanding of Eilat's current transportation network
    
    \item Create predictions for Eilat's future transportation situation in 2040
    
    \item Produce a roadmap that guides Eilat towards transportation in 2040.
\end{itemize}

\newpage
\section{Background}

\subsection{Existing Public Transport}[Christopher M., Zachary Z.]
\begin{itemize}
    \item Current bus situation
    \item Small transport \& taxis/ride sharing
    \item Route 90
    \item Tourism impact
\end{itemize}

% Probably needs to be integrated into the main document a little better
Israel's route 90 is the main highway leading into Eilat on its northern edge and runs parallel to the current Eilat Airport, which (as detailed below) is scheduled to shut down as the new Ramon airport opens, and Ovda airport is expected to cease handling civilian flights. This leaves Ramon airport as the only airport servicing Eilat, and therefore leaves Route 90 as the only road handling its traffic. However, route 90 does not as of yet appear to be set up to handle such traffic, with only a simple road junction for the airport and a series of traffic circles in the city itself, instead of a traditional highway on-ramp/exit system to allow unimpeded traffic flow. With traffic congestion in Eilat near the route's entrance to the city and a large influx of tourists from the new airport, heavy congestion and increased travel times can be expected for tourists, residents, and businesses alike.

\subsection{Airports near Eilat}[Evan G.]
\begin{itemize}
    \item Passenger statistics
    \item New Airport passenger expectations
    \item people per hour for trains
\end{itemize}

\subsection{Trains \& Trams}[Evan G., Christopher M.]
\begin{itemize}
    \item Train to Ramon airport?
		\begin{itemize}
			\item Potential types of trains and their respective costs, speeds, and environmental impact
    	\end{itemize}
    \item Intra-city trams, pros \& cons
    \item Case examples of other city's tram systems
\end{itemize}

% Evan's thing on trains here?

One option for transport within Eilat is a tram system. Briefly, trams are small rail-based vehicles that run along tracks either purpose-laid in designated areas, or through roads for purposes of crossing or even for long distances so the tram can act as a street vehicle. Trams combine the capacity and efficiency of a train with the accessibility and cost of a bus. Trams can stop at any point and take on passengers without need for a station and, since they run on tracks and therefore can't move back and forth within a lane, are an efficient use of space on or off a road. Trams are typically all-electric, with power delivered by one or two overhead lines strung well above the tracks, making trams both energy efficient and zero-emission. If a train line to the new airport was established, the tram network could theoretically be connected to that, too, making trams an easy connection both within and outside the city. Tram systems are also well tested, with hundreds across the globe (including one in Jerusalem) carrying billions of passengers per year.

Tram systems can be challenging to set up, so some intermediary steps or developments may be needed. Trams require rails to run on, posing an up-front infrastructure cost beyond the cost of vehicles alone. High-voltage wires to deliver power are also needed but are likely far cheaper than embedding rails in roads, so an intermediary system of trolleybuses (electric buses powered by overhead lines) may be useful to ease the transition and ensure its viability before completely committing to it. 



\subsection{Smart Cities}[Valeria K.]
\begin{itemize}

\item What are smart cities?
    \begin{itemize}
        \item Overall picture
    \end{itemize}
\item Smart traffic management
    \begin{itemize}
        \item Monitor and analyze traffic flow
        \item Waze
    \end{itemize}
\item Smart meter
    \begin{itemize}
        \item App that let users know where they can find free parking spaces
        \item Digital payment
     \end{itemize}
\item Smart public transit
    \begin{itemize}
        \item Fulfill riders needs real time
    \end{itemize}
        \begin{itemize}
            \item Improves efficiency and satisfaction
        \end{itemize}
\item Energy conservation
    \begin{itemize}
        \item Smart sensors: dim/turn off street lights when there's not pedestrian
           \end{itemize}
\item Collaboration between private and public sector
     \begin{itemize}
         \item Transparency- making collected data available to citizens
     \end{itemize}
\end{itemize}



\newpage
\section{Methods}


% Bibliography -- automatically managed in APA style, on a new page
\newpage
\bibliographystyle{apacite}
\bibliography{refs}

\end{document}
