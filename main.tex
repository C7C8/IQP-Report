\documentclass[12pt]{article}                         % Make it an article
\usepackage{indentfirst}                       % All paragraphs indented
\usepackage{setspace}                           % For 1.5 spacing
\usepackage[hidelinks]{hyperref}      % Make table of contents clickable
\usepackage[margin=1in]{geometry} % 1 inch margin
\usepackage[english]{babel}                     % For "fancy" quotes
\usepackage[autostyle, english=american]{csquotes} % Same
\usepackage{apacite}                            % Use APA citations
\usepackage{etoolbox}                           % Misc
\usepackage{titlesec}
\usepackage{listofauthorships}
\MakeOuterQuote{"}                              % For fancy quotes
\usepackage{times}                              %For Tilde
\usepackage{textcomp}                           %For Tilde
\titlespacing*{\section}{0pt}{0pt}{0pt}
\titlespacing*{\subsection}{0pt}{0pt}{0pt}
\titlespacing*{\subsubsection}{0pt}{5pt}{-3pt}

\title{Eilat Transportation}
\author{Evan Goldstein, Valeria Kopper, Christopher Myers, Zachary Zlotnick}
\date{January 2018}

\begin{document}
\pagenumbering{gobble}
\maketitle
\newpage

\renewcommand\abstractname{Summary} % "Abstract" -> "Summary", the header lies

\pagenumbering{roman}
\tableofcontents
\newpage
\listofauthorships
\newpage
\pagenumbering{arabic}
\doublespacing

% Actual content begins here!
\section{Introduction}

Goal: The goal of this project is to aid the city of Eilat in investigating possible transportation services that will minimize environmental impact, congestion, and travel times for residents, tourists, and businesses while preserving travel friendliness and experience.

Objectives:
\begin{itemize}
    \item Develop an understanding of Eilat's current transportation network
    
    \item Create predictions for Eilat's future transportation situation in 2040
    
    \item Produce a roadmap that guides Eilat towards transportation in 2040.
\end{itemize}

\newpage
\section{Background}

\subsection{Existing Transportation}[Christopher M., Zachary Z.]
Israel's route 90 is the main highway leading into Eilat, positioned on its northern edge and running parallel to the current Eilat Airport. The Ovda Airport is scheduled to shut down as the new Ramon Airport opens alongside route 90, and Ovda Airport is expected to cease handling civilian flights. This leaves Ramon airport as the only airport servicing Eilat, and therefore route 90 as the only road handling tourist traffic. Route 90 does not appear set up for this, with only a simple road junction for the airport and a series of traffic circles in the city itself. With already present congestion in Eilat near the route's entrance to the city and tourist traffic from the new airport, overall heavy congestion and increased travel times are expected.

Alternatives to conventional public methods of transportation already exist to bridge the gaps in commuting. "Bird" has successfully implemented an electric scooter system in over 100 cities worldwide -- solving public transport's "last mile" problem. Riders pick up a scooter from any "nest" in the city and pay an initial fee plus a fixed fee for each minute used \cite{BirdIsrael}. Bird riders usually don't exceed 3km per use, suggesting short-range use only, for example to or from bus stops. \cite{MichaelRaz-Chaimovich2018BirdGlobes}. Scooters  in Tel Aviv are used heavily to avoid difficult road traffic. Another alternative is Mobike, which provides similar ease-of-access bike transportation, but at much lower cost, both in fees and maintenance. Finally, another option is CAR2GO, which allows users to rent a car for any time, but without requiring human intervention or paper signing. The cost of parking and insurance is part of the rental fee, and the experience is fairly seamless for users \cite{OrenDoriandMeiravMoran2018IsraeliHaaretz.com}. CAR2GO is used in 6 Israeli cities and could be used in Eilat, appealing to tourists with its ease of use.

\subsection{Airports near Eilat}[Evan G.]
In 2017, 1.4 million passengers passed through the J. Hozman airport in Eilat, and another 200 thousand passed through Ovda International Airport for a total of 1.6 million passengers between the two airports. Ramon Airport will replace both the Eilat and Ovda airports, handling both domestic and international flights. Ramon will accommodate up to 2 million passengers a year, and 4.2 million passengers a year by 2030. This implies a minimum of over 600 passengers per hour, assuming flights landing between 6am and 12:00am (an 18 hour schedule). Transporting this many to and from the city will require a transportation method that minimizes congestion, environmental impact, and travel time.

\subsection{Trains \& Trams}[Evan G., Christopher M.]
Trains are an option for minimizing congestion, as they would run on dedicated tracks instead of roads. A rail line between the airport and the city could be integrated with the existing line between Tel Aviv and Jerusalem. If speed is important, maglev trains are the fastest type available, but also the most expensive, costing between 50 and 200 million dollars per mile. Light rail is the most common type of rail transit used for short-range, often intra-city, transport, but typically average between 20-30 mph. Metrail, a hyrbid solution, is a lower cost and environmentally friendly option, and can be installed for only 20 million dollars per mile. A compromise between cost, speed and environmental impact is a more traditional rail system with trains such as Bombardier's TALENT 3, which has a cost of less than 7.5 million dollars per train car and speeds of 160 kph. The TALENT 3 uses traditional rails, which have an installation cost of between 1 and 2 million dollars per mile.

Another option for transportation within Eilat is a tram system. Trams are small rail-based vehicles that run along tracks in designated areas, or through roads for crossing, allowing the tram to act as a street vehicle. Trams combine the capacity and efficiency of a train with the accessibility and cost per unit of a bus. Trams can stop at any point and take on passengers without a station and are an efficient use of space on or off a road. Trams are typically all-electric, with power delivered by one or two overhead lines strung well above the tracks, making trams both energy efficient and zero-emission. If a train line to the new airport was established, the tram network could theoretically be connected to that too, making trams an easy connection both within and outside the city. Tram systems have been well-tested, with hundreds across the globe carrying billions of passengers per year, including one in Jerusalem.

Tram systems can be challenging to set up, so some intermediary steps may be advised. Trams require rails to run on, posing an up-front infrastructure cost beyond the cost of vehicles alone. High-voltage wires to deliver power are also needed but are likely far cheaper than embedding rails in roads, so an intermediary system of trolleybuses (electric buses powered by overhead lines) may be useful to ease the transition and ensure its viability before completely committing to it. 

\subsection{Smart Cities}[Valeria K.]

Through the use of innovative technologies that incorporate data analysis, cities have begun to develop an interconnected web, increasingly through the Internet of Things (IoT), that allows for maximized city function efficiency \cite{MargaretRouseSmartCity}. These are known as smart cities, which can lead to economic growth and better life quality for citizens. Successful smart cities tend to have strong support from their government, especially in regards to open data platforms and smart sensors, as well as collaboration between the private and the public sectors \cite{BrianZanghi2017WhyExamples}. Oftentimes transparency, achieved by making collected data available to citizens, greatly increases confidence, which plays a role in the successful development of smart cities.

A combination of technologies, such as sensors throughout the city, can lead to improved traffic flows with decreased traffic jams. Smart traffic management is key to this. Through monitoring and analyzing traffic flow cities can better adjust and manage public transportation and traffic lights. Waze, a navigation app, has partnered with several cities in order to form a two-way exchange of information called the Connected Citizens Program \cite{Stern2016WazeMobility}. Other cities have implemented smart meters, which show users in an app where they can find free parking spaces; it also allows for digital payment. Furthermore, existing smart cities have found that smart public transit improves the efficiency and the satisfaction of users by fulfilling the real time demand. Through smart sensors there can be a significant energy conservation since street lights can be dimmed when there are no pedestrians in the area. Overall, the technology found in smart cities improves life quality. 

\newpage
\section{Methods}

*observational research on current congestion, methods of transportation (especially airport), and understanding the effects of tourism (although we are at a disadvantage due to the time we're here)

*Do some computery things to generate graphs and diagrams of Eilat's traffic situation (Waze, google maps?, etc)

*Obtain statistics from Eilat regarding city growth and development over recent years to predict future population and transportation development

*Investigate possibilities for train stations and tram rails/wires

*Look at tourist data and Eilat publicity to estimate tourism increases (e.g. through google search trends)

% Bibliography -- automatically managed in APA style, on a new page
\newpage
\bibliographystyle{apacite}
\bibliography{refs}

\end{document}
