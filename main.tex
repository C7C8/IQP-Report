\documentclass[12pt]{article}                         % Make it an article
\usepackage{indentfirst}                       % All paragraphs indented
\usepackage{setspace}                           % For 1.5 spacing
\usepackage[hidelinks]{hyperref}      % Make table of contents clickable
\usepackage[margin=1in]{geometry} % 1 inch margin
\usepackage[english]{babel}                     % For "fancy" quotes
\usepackage[autostyle, english=american]{csquotes} % Same
\usepackage{apacite}                            % Use APA citations
\usepackage{etoolbox}                           % Misc
\usepackage{titlesec}
\usepackage{listofauthorships}
\MakeOuterQuote{"}                              % For fancy quotes
\usepackage{times}                              %For Tilde
\usepackage{textcomp}                           %For Tilde
\titlespacing*{\section}{0pt}{0pt}{0pt}
\titlespacing*{\subsection}{0pt}{0pt}{0pt}
\titlespacing*{\subsubsection}{0pt}{5pt}{-3pt}

\title{Eilat Transportation}
\author{Evan Goldstein, Valeria Kopper, Christopher Myers, Zachary Zlotnick}
\date{January 2018}

\begin{document}
\pagenumbering{gobble}
\maketitle
\newpage

\renewcommand\abstractname{Summary} % "Abstract" -> "Summary", the header lies

\pagenumbering{roman}
\tableofcontents
\newpage
\listofauthorships
\newpage
\pagenumbering{arabic}
\doublespacing

% Actual content begins here!
\section{Introduction}

Goal: The goal of this project is to aid the city of Eilat in investigating possible transportation services that will minimize environmental impact, congestion, and travel times for residents, tourists, and businesses while preserving travel friendliness and experience.

Objectives:
\begin{itemize}
    \item Develop an understanding of Eilat's current transportation network
    
    \item Create predictions for Eilat's future transportation situation in 2040
    
    \item Produce a roadmap that guides Eilat towards transportation in 2040.
\end{itemize}

\newpage
\section{Background}

\subsection{Existing Public Transport}[Christopher M., Zachary Z.]
Israel's route 90 is the main highway leading into Eilat, positioned on its northern edge and running parallel to the current Eilat Airport. Eilat Airport (as detailed below) is scheduled to shut down as the new Ramon airport opens alongside route 90, and Ovda airport is expected to cease handling civilian flights. This leaves Ramon airport as the only airport servicing Eilat, and therefore leaves route 90 as the only road handling its traffic. However, route 90 does appear set up to handle such traffic, with only a simple road junction for the airport and a series of traffic circles in the city itself, instead of a traditional highway on-ramp/exit system to allow unimpeded traffic flow. With traffic congestion in Eilat near the route's entrance to the city and a large influx of tourists from the new airport, heavy congestion and increased travel times can be expected for tourists, residents, and businesses alike.

Alternatives to conventional public methods of transportation already exist to bridge the gaps in one's commute. Bird has successfully implemented an electric scooter system in over 100 cities worldwide - solving the "last mile" problem that exists within public transportation. The way the system works is riders will pick up a scooter from any "nest" in the city and pay an initial fee (usually 1-2 dollars), and a fixed fee for each minute used. "Chargers" are people who will collect the scooters from the street and charge them in their homes for a rebate \cite{BirdIsrael}. Statistics show that Bird riders usually don't exceed about 3km on their average use, suggesting that most people are not using these scooters to get directly from their residence to their final destination (*). Rather, they are most likely using these scooters to get from the bus stop to work, or vice versa. The scooters are already in Tel Aviv and, from first-hand observation, are being leaned on heavily to avoid the messy car situation of the city. Another alternative is Mobike, which provides a similar ease-of-access method of transportation as the Bird scooter. However, they are extremely inexpensive (around 25 cents per day) since they require no charging. They are also very durable and easy to maintain, and don't necessarily need to be dropped at a certain site, allowing for more direct commutes (*). Overall, these two players in the global transportation scheme use the niche strategy to address the issue of city-wide congestion, and would alleviate stressors under peak hours if properly implemented. 

Another niche option for commuting within Israel is CAR2GO, which offers users the ability to effectively rent a car for whatever time they desire. It differentiates itself from other major car rental companies by leaving cars in specific parking spaces for consumers to then pickup and use, skipping the hassle of going to a clerk and signing papers. The cost of parking and insurance is already priced into the rental fees, and the experience is fairly seamless for users \cite{OrenDoriandMeiravMoran2018IsraeliHaaretz.com}. They are in use currently in 6 different Israeli cities and could make its way into Eilat, where it would most likely garner the attention of the constant influx of tourists due to its ease-of-access. 

\subsection{Airports near Eilat}[Evan G.]
In 2017, 1.4 million passengers passed through the J. Hozman airport in Eilat, and another 200 thousand passed through Ovda International Airport for a total of 1.6 million passengers between the two airports. Upon opening, the Ramon Airport will replace both the Eilat and Ovda airports, handling both domestic and international flights. Initially, Ramon will be able to accommodate up to 2 million passengers a year, expanding to accommodate 4.2 million passengers a year by 2030.

4.2 million passengers per year is a minimum of over 600 passengers per hour, assuming flights landing between 6am and 12:00am (an 18 hour schedule). Transporting 4.2 million passengers per year into and out of the city will require a transportation method that minimizes congestion, minimizes environmental impact, and, ideally, minimizes travel time. Trains and buses are two good solutions to this problem, both with associated pros and cons.

\subsection{Trains \& Trams}[Evan G., Christopher M.]
Trains are an option for minimizing congestion, as they would run on dedicated tracks instead of roads. Additionally, a rail line between the airport and the city could be integrated with the existing line between Tel Aviv and Jerusalem. Many types of trains exist. Maglev trains are the fastest type available, and is also the most expensive, costing between 50 and 200 million dollars per mile. Light Rail is the most common type of rail transit used for short-range (often intra-city) public transit, but typically average between 20 and 30 mph (though some are capable of reaching highway speeds). There also exist lower cost and very environmentally friendly options such as Metrail, a hybrid solution which can be installed for only 20 million dollars per mile. A compromise between cost, speed and environmental impact are more traditional rail systems with trains such as Bombardier's TALENT 3, which has a cost of less than 7.5 million dollars per train car, and can reach speeds of 160 kph. The TALENT 3 uses traditional rails, which have an installation cost of between 1 and 2 million dollars a mile.

One option for transport within Eilat is a tram system. Briefly, trams are small rail-based vehicles that run along tracks either purpose-laid in designated areas, or through roads for purposes of crossing or even for long distances so the tram can act as a street vehicle. Trams combine the capacity and efficiency of a train with the accessibility and cost of a bus. Trams can stop at any point and take on passengers without need for a station and, since they run on tracks and therefore can't move back and forth within a lane, are an efficient use of space on or off a road. Trams are typically all-electric, with power delivered by one or two overhead lines strung well above the tracks, making trams both energy efficient and zero-emission. If a train line to the new airport was established, the tram network could theoretically be connected to that, too, making trams an easy connection both within and outside the city. Tram systems are also well tested, with hundreds across the globe (including one in Jerusalem) carrying billions of passengers per year.

Tram systems can be challenging to set up, so some intermediary steps or developments may be needed. Trams require rails to run on, posing an up-front infrastructure cost beyond the cost of vehicles alone. High-voltage wires to deliver power are also needed but are likely far cheaper than embedding rails in roads, so an intermediary system of trolleybuses (electric buses powered by overhead lines) may be useful to ease the transition and ensure its viability before completely committing to it. 

\subsection{Smart Cities}[Valeria K.]

Through the use of innovative technologies that incorporate data analysis, cities have begun to develop an interconnected web, increasingly through the Internet of Things (IoT), that allows for maximized city function efficiency \cite{MargaretRouseSmartCity}. These are known as smart cities, which can lead to economic growth and better life quality for citizens. Successful smart cities tend to have strong support from their government, especially in regards to open data platforms and smart sensors, as well as collaboration between the private and the public sectors \cite{BrianZanghi2017WhyExamples}. Oftentimes transparency, achieved by making collected data available to citizens, greatly increases confidence, which plays a role in the successful development of smart cities.

A combination of technologies, such as sensors throughout the city, can lead to improved traffic flows with decreased traffic jams. Smart traffic management is key to this. Through monitoring and analyzing traffic flow cities can better adjust and manage public transportation and traffic lights. Waze, a navigation app, has partnered with several cities in order to form a two-way exchange of information called the Connected Citizens Program \cite{Stern2016WazeMobility}. Other cities have implemented smart meters, which show users in an app where they can find free parking spaces; it also allows for digital payment. Furthermore, existing smart cities have found that smart public transit improves the efficiency and the satisfaction of users by fulfilling the real time demand. Through smart sensors there can be a significant energy conservation since street lights can be dimmed when there are no pedestrians in the area. Overall, the technology found in smart cities improves life quality. 

\newpage
\section{Methods}


% Bibliography -- automatically managed in APA style, on a new page
\newpage
\bibliographystyle{apacite}
\bibliography{refs}

\end{document}
